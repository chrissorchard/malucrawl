\Section{Related work}
deSEO: Combating Search-Result Poisoning(academic paper)
Over the past few years, the internet has been used by an attacker to distribute a malware. Search engines can be used as a medium for spreading a malware through the internet. Essentially, attackers use the search engine optimisation algorithm applied by Google to insert their malicious links in the most visited websites. Based on deSEO observations, attackers used trendy keywords and regenerated relevant content pages in order to achieve a successful attack. 
Therefore, a recent study has been undertaken to detect the malicious links based on such trendy keywords. In this study, the proposed detection method identifies suspicious websites by analysing the components of the URLs. In the second step, each suspicious websites is clustered based upon its lexical structures. Finally, a cluster analysis is undertaken in order to select the suspicious group.  
The dataset required for this experiment was collected from Google and Bing search engines. The data set was divided into two groups, the first group was used for training and the second group was used for the purpose of learning. The system can automatically identify additional malicious URLs. However, detection will most probably fail if the attacker does not use the keyword in the URL. 

Fashion Crimes: Trending-Term Exploitation on the Web (academic paper)
A recent study undertaken in 2011, analyzed the increasing amount of malware that related to the trendy-terms. This research provides the first large-scale measurements of using trending terms in web search engines over a period of nine months. This research how the attackers use trending keywords to direct users to advertising websites. These websites profit by displaying advertisements on their webpage without offering any goods or services. \cite{moore2011fashion}
The proposed methodology is based on data collection and website classification. The data for this study is collected from different sources such as Google, Bing and Twitter. The websites are classified into two groups, malicious and benign. The malicious websites can be a malware hosting website, or may contain advertisements.\cite{moore2011fashion}
According to this paper, their system added hot trendy topics to the list of trendy term every hour and removed older terms from the list after 14 days. The result of their measurements showed that a high percentage of the malware belonged to Google Hot trending topics. Also, the comparison between trendy terms and popular topics highlighted that both topics were valuable tools for attackers in distributing the malware. However, trendy terms returned a higher number of infected webpages. The difference being that a popular topic can last for a longer period of time and its related website always has visitors, whilst a trendy term will stay popular as a ‘hot topic’ for a short period of time and have a lot of visitors during that short time period.\cite{moore2011fashion}
