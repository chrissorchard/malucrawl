\section{Introduction}

\subsection{Problem Definition}

\paragraph{} 
The continually increasing variety of known and unknown malwares make malware detection a difficult problem to solve. As malware detection systems become more intelligent and sophisticated, malware writers are also developing more advanced methods. Some anti-virus software is ineffective in terms of detecting unknown malware. This project seeks to design a malware detection system capable of identifying web pages containing malicious content for a given contextual topic as represented by trending search keywords. 

\paragraph{} 
Also, this research involved emulating a user browsing for these keywords. The challenging part of this project arose in seeking to define a malicious web page, as there are many variations of an infected webpage depending on the type of malware. 
Some types of malware directs the user to an illegal website by clicking on the URL. A large group of malware are designed with a commercial objective in mind. A webpage that has poor content, offers no goods or services and is used for the purpose of directing users to unsolicited advertisements is an example of commercial malware. Whilst other web pages may look perfectly fine at first glance but in fact contain malicious functions, for example by using JavaScript functions. In the case of the latter, the detection is even more difficult because the malware is a type of obfuscation malware that is hidden from the user. Moreover, malwares could come from a variety of different sources.
The examples above highlight the difficulties of this project. Furthermore, whilst a great deal of research has been performed on the subject of malware detection and the effectiveness of various detection methods, such research specifically considered a specifically defined type of malware and analysed the methods used for that particular malware. 

\paragraph{} 
In seeking to propose a malware detection system capable of identifying web pages containing malicious content based on trending search keywords, this paper gives consideration to the fact that different variations of malware exist. 
