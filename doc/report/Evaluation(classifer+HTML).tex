/Section{Evaluation}
/Subsection{Calssifier}
The classification algorithm can detect unknown URLs with the aid of a pattern matching algorithm. The result of the classifier will grade the maliciousness of the given URL with a confidence rate. However, the pattern-matching algorithm is not very fast. The main reason is for any given URL, the classifier has to check the pattern of all URLs in the database to find if there are any matches. After running the system each time, five malware scanners update the database with their returned result. Therefore, the list of URLs in the database increases after each update. As a result, over time the classifier has to undertake more pattern matches for a given unknown URL, which in turn results in a decreasing processing speed over time.
According to the many studies performed on the subject of malware detection, applying machine learning algorithms when implementing a classifier will result in an efficient detection system. However, the requirements for employing sophisticated machine learning techniques is having a dataset in advance to able learning the detection machine for the purpose of the classification which did not meet the needs of this project. Because, this is a time limited project and the dataset produced by scanners very close to the deadline of project. A complete alternative plan explained in the future developments section. 
/Subsection{HTML malware scanner}

The design of HTML malware scanner is based on the frequency of reappearing trendy-terms within the web page of a given URL. Also, the scanner searches all associated webpage hyperlinked within the given URL. The main advantage of the HTML malware scanner is definition independency, which means the design is independent of the definition of a malicious web page. Also the employed method observed the relation between using trendy-term in the search engine and finding the malicious web page.
