\Section{Future development}
The main problem about classier is that it is not fast. With the aid of the machine learning algorithm the URL classification will be faster. There is a list of URLs in the database some of which are malicious and others are benign. Malicious URLs can represent the form of a vector and also clean URLs can convert to a vector. Having two vectors, allows us to apply Support Vector Machine (SVM) or Multi-Layers Perceptron (MLP). At the moment, the pattern matching algorithm needs to match the given URL with every single URL in the database. However, by having two vectors the number of matches will reduce noticeably. Converting the URL to a form of a vector is very challenging because the structure of the URL does not have repetitive pattern. 
The second suggestion for the classifier is to use the different classification algorithms and then compare and contrasts their performance. Several machine learning techniques such as Naïve Bayes, Support Vector Machine (SVM) and Decision Tree can be used for the implementation. The list of URLs can be used as a training data for the classifier; also another list of URLs should be used for the testing purpose. URLs in the testing dataset should be completely different from those URLs in the training dataset. The important part is choosing features of URLs for the classification. Lots of features can be considered such as, length of the URL, number of using “/”or number of using “.” in the URL structure. The input for the classifier is features extracted from the URL’s structure. A classifier will use extracted features to distinguish a malicious URL from the benign URL. 
In the current system the HTML malware scanner accept an URL and a keyword as an input based on the frequency of the keyword in the web page of the given URL decides whether the URL is malicious or clean. Also, as it explained in the section?, the first attempt to design the HTML scanner was having the set of trendy keywords saved in our database. Therefore, the HTML scanner just takes an URL as an input and updates the database with the new coming trendy-term. The challenging part of this model is the design of the database. Database contains a set of trendy-terms and each trendy-term belongs to a group of keywords that have similar context, (see the example in the section ?). The database will be connected to a list of selected trusted website such as BBC News. Therefore, the database can be updated after any new trendy-term coming to the News. And then the old terms can be removed from the database after for example four weeks. 
In the model, the scanner decision is not only based on the one trendy-keyword, it is based on the group of words connected to the trendy-term in the database as well. Therefore the result of the malware scanner will be more reliable.
Another strategy for design a HTML malware scanner is considering the features of the web page related to the each URL. In this method, a list of features of the webpages extracted. This method is very similar to the one that has suggested for the classifier earlier in this section. There is a list of training data set and testing data set for the classifier, but extracted features from web page will used as an input for the classifier. The web page features can be the number of hyperlinks in the webpage, number of lines, word count and functions (JavaScript keywords).
After training the classifier, some feature shows more important result from other ones for example use of JavaScript function in the web page. Hence, those features will be selected for classification to produce more accurate result. Although this model has not observed the relation between the trendy-term and malicious URLs. But by combining the above two models, we have a powerful system that have a updated database and a classifier. The database contains the list of trendy-terms updated from trusted websites and the classifier that capable to identify unknown malicious URLs that related to the trendy-terms.
The malware detection system that employs machine-learning techniques has a few advantages. The system can overcome the problem of detecting an unknown malware and also polymorphic malwares. In addition, it is very efficient and resilient in comparison with other malware detection techniques. Moreover, the adapted techniques in this system make it difficult for malware writers to bypass the detection system. 
