
\subsection{Calssifier}

The classification algorithm can detect unknown URLs with the aid of a pattern matching algorithm. The result of the classifier will grade the maliciousness of the given URL with a confidence rate. However, the pattern-matching algorithm is not very fast. The main reason is for any given URL, the classifier has to check the pattern of all URLs in the database to find if there are any matches. After running the system each time, five malware scanners update the database with their returned result. Therefore, the list of URLs in the database increases after each update. As a result, over time the classifier has to undertake more pattern matches for a given unknown URL, which in turn results in decreasing processing speed over time.

\paragraph{} 
According to the many studies performed on the subject of malware detection, applying machine learning algorithms when implementing a classifier will result in an efficient detection system. Employing sophisticated machine learning techniques requires a dataset in advance to enable machine learning to occur for the purpose of classification. However, the latter requirement cannot be fulfilled because this project is time limited.

