\section{High interaction malware detection with Wine}
\subsection{Design}
This section details one of the high interaction malware detection 
approaches used in the system. High interaction requires completely functional systems 
and realistic user interaction with the target malicious websites, which means 
it consumes more system resources and demands a more complicated 
framework compared to low interaction and no interaction methods. The major 
feature of this approach is capability to detect unknown malicious 
actions. This project has a system called Wine Explorer, 
which is typical of a high interaction approach. It uses Wine as a compatibility layer to run 
Internet Explorer 6 in a Linux environment and is capable of scanning for 
possible file operations performed by malicious websites. 

\subsubsection{Wine Is Not an Emulator}
Wine is a widely used free software which acts as a compatibility layer 
capable of running Windows applications on non-Windows operating 
systems.\cite{wikiwine} It us unlike virtual machines and emulators which simulate 
internal Windows logic, ``Wine translates Windows API calls into POSIX calls 
on-the-fly, eliminating the performance and memory penalties of other methods 
and allowing you to cleanly integrate Windows applications into your 
desktop.''\cite{aboutwine} Originally a ``backcronym'' WINE stood for ``Wine Is Not an 
Emulator''. Wine is able to run most Windows applications 
without any performance drop as long as there are no performance related bugs, 
especially for 2D applications. Whilst Wine is an 
extra layer on the top of the system, there is no difference to programs that 
uses extra libraries. An example of such software is the compatibility 
mode in newer versions of Windows designed for running legacy 
software.\cite{wineperformance}

Wine has advantages over conventional virtual machines particularly in terms 
of working with web browsers. Although 
virtual machines have advantages in simulating programs which give more 
realistic environments, they use additional resources. In most cases, a virtual machine consumes significantly more memory, disk space and 
CPU, as it needs to simulate the whole operating system instead of a providing a layer to a program. Wine treats Windows applications as first-class 
citizens, which run at full speed.\cite{wineperformance}

Wine uses a working directory as a complete Windows system which stores all 
information about the specific system's configurations and files in its virtual 
hard drive. Moreover, the active working directory is known as the Wine prefix. 
One of our discoveries about Wine is that the creation of Wine prefix is 
lightweight and resource saving compared to that of a virtual machine or
emulator, while the Wine prefix itself also conforms to the nature of a 
sandbox to some extent, which means almost all kind of behaviours performed 
by programs inside the Wine prefix cannot affect the underlying system. 
Although there is a potential threat that is the unrestricted permissions 
for applications in Wine prefix, which by default exposes the root file 
system to them, we still conclude Wine is a suitable tool for creating a 
client honeypot in high interaction malware detection. \\

\subsubsection{Execution Flow}
Wine Explorer is a straightforward program, and the following graph 
illustrates the execution flow of detecting a single URL. \\
% Flow chart here
With Wine Explorer we open the URLs given by the URL classifier (introduced 
previously) with the specific instance of Internet Explorer inside the Wine 
prefix. 
Those URLs are said to be strongly suspecious of containing malware and are 
supposed to be investigated in depth. 
In order to speed up Wine prefix creation, we pack a clean Wine prefix with 
Internet Explorer 6 installed. Therefore whenever a new Wine prefix is needed, 
we just unpack it to a desired location. 
The package's SHA-256 hash is calculated and stored in Wine Explorer such that 
its validity can be ensured. 
The package is then uploaded into a web drive and can be downloaded in 
order to increase the system's portability. \\
The system should also be able to store multiple instances of Wine prefixes 
and execute Internet Explorers in them concurrently. 
After an execution is completed a scan is performed to check file system 
modifications inside that Wine prefix. 
Due to various reasons the file system might have changes performed by normal 
operations such as cache files. 
Therefore at this step we provide a white list that describes the kind of 
files that should be ignored, which is applied to filter the results. 
The results are then returned to the main system and the temporary Wine 
prefixes after scanning should be removed in order to save disk space. 

\subsection{Implementation}
We use Python to implement this program just as other parts of our project. 
In the program an instance of Wine prefix running Internet Explorer, which is called 
a WineBrowser, is a threaded class manages all operations with a single 
given URL. This is how the program work in details:
\begin{enumerate}
\item \verb`create_wine_prefix(template_spec)` \\
Check cache for the package containing the predefined Wine prefix, if not 
exist download it, then perform a hash check.
If success the package is unpacked to a temporary directory. 
In the file system there should also be a clean Wine prefix unpacked from the 
package, which is used for checking file system changes later. 
\item \verb`WineBrowser.__init__()` and \verb`WineBrowser.run()`\\
The active Wine prefix is set to the temporary directory, and the Internet 
Explorer is run with the given URL under Wine. 
\item
Wait for 30 seconds.
\item \verb`check_change(dir1,dir2)`\\
Check the file system changes. This is achieved by running {\em filecmp} recursively 
for all files between the clean Wine prefix and the active one. The results 
are filtered against a white list then returned to the main system. The white 
list is included in the appendix TODO. 
\item \verb`WineBrowser.close()`\\
The temporary directory is removed. 
\end{enumerate}
We choose 30 seconds as the waiting interval is because of possible network 
latency and long computing time. Any webpages take more than 30 seconds to 
load will be discarded as if not they will significantly lower the system's 
throughput. \\
The main system runs WineBrowser instances with Celery, one per task, in order 
to achieve concurrent execution of multiple Wine programs at the same time, 
which increases the overall throughput dramatically. 
