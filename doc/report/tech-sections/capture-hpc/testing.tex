\subsection{Testing}

Due to the complexity of the Capture-HPC malware scanner, the individual
components that make up the scanner as seen by the framework had to be first
tested individually, and then tested as a single component.

The first unit tests involved running Capture-HPC from the command line, and it
was quickly discovered that some work was needed to get the revert script to
function correctly. After spending a considerable amount of time getting an RPC
mechanism working, it was possible to test Capture-HPC on some URLs. The URLs
used for unit testing consisted a small list of very common websites, and an
executable hosted on an ECS internal server. When configured to detect any
downloaded executable as malware, Capture-HPC correctly identified the
executable file as malware.

The framework integration was unit tested by running the integration script
manually in a django shell, and it was discovered that another RPC comonent was
needed as the version of Python on the gateway was not sufficient. Once the RPC
component had been completed and the integration script proven to work,
Capture-HPC was tested in the full framework, and after revealing a few bugs in
the framework it was successfully running on URL lists. Further testing will
need to be carried out to determine whether the exclusion lists are correct,
with any missing parts coming to light over time as the framework is used.
