\section{Project Brief}
%DNS\nomenclature{DNS}{Domain Name Service}

\newpage
\mbox{}
\newpage
\mbox{}
\newpage

\section{List of Abbreviations}
\renewcommand{\nomname}{}
\printnomenclature

\clearpage
\section{User Manual}
\label{sec:manual}


\clearpage
\section{Contents of Source Disk}

Each directory has a README files describing its contents.

\subsection{Directories}
\begin{enumerate}
\item\texttt{db-tools/} - Node database tools and Icinga Templates
\begin{enumerate}
 \item\texttt{bottle.py} - copy of Bottle.py web framework.
 \item\texttt{list-serve.py} - normal python version of list-serve using default sever.
 \item\texttt{list-serve.wsgi} - wsgi compatible version of the server.
 \item\texttt{templates/} - templates used for Icinga config generation.
 \item\texttt{update-icinga.py} - script for taking values from the node database, and creating
Icinga configuration files.
 \item\texttt{upload-db.py} - script for adding new nodes to the database.
\end{enumerate}
\item\texttt{edumond-node/} - Configuration running on the wireless node
\begin{enumerate}
 \item\texttt{lighttpd/} - copy of /etc/lighttpd, containing lighttpd.conf with fastcgi
 \item\texttt{lua/} - copy of /usr/lib/lua, containing json.lua and rad-check.lua
 \item\texttt{rad-check/} - copy of /etc/rad-check, containing config.lua and client and server keys/certs
 \item\texttt{magnet} - binary used by fastcgi to run Lua scripts with POST data.
 \item\texttt{packages} - list of openwrt packages installed
 \item\texttt{send\_nsca.cfg} - copy of /etc/send\_nsca.conf, configured to
contact Icinga server
\end{enumerate}
\item\texttt{kanga-config/} - Configuration for Icinga, FreeRADIUS, and PNP4Nagios
\begin{enumerate}
 \item\texttt{www/} - copy of /var/www, node serving code in www/wsgi
 \item\texttt{httpd/} - copy of /etc/httpd, nodes server set up in httpd/conf/httpd.conf
 \item\texttt{raddb/} - copy of /etc/raddb, FreeRadius configuration files.
 \item\texttt{nagios/} - copy of /etc/nagios, contains NSCA configuration in nagios/nsca.cfg
 \item\texttt{icinga/} - copy of /etc/icinga, Icinga configuration files
\begin{enumerate}
                \item icinga/objects/matrix-hosts - test device defenitions
                \item icinga/objects/matrix.cfg - general test configuration
                \item icinga/objects/commands.cfg - contains added commands
                \item icinga/objects/templates.cfg - contains service templates
\end{enumerate}
 \item\texttt{nodedb.sql} - Dump of the MySQL database of the nodes
\end{enumerate}
\item\texttt{kangab-config/} - Configuration for second FreeRADIUS server.
\begin{enumerate}
 \item\texttt{raddb/} - copy of /etc/raddb, FreeRadius configuration files.
\begin{enumerate}
                \item raddb/clients.conf - clients with secrets
                \item raddb/proxy.conf - realm configuration
\end{enumerate}
\end{enumerate}
\item\texttt{rad-check/} - Test data generation code
\begin{enumerate}
 \item\texttt{credcheck.lua} - fastcgi response script
 \item\texttt{eapol\_test} - cross-compiled eapol\_test tool
 \item\texttt{edumond.lua} - Lua credential exchange and testing script
 \item\texttt{edumond.py} - deprecated python version
 \item\texttt{index.lua} - debug script
 \item\texttt{json-LICENCE.txt} - License for Lua JSON library
 \item\texttt{json.lua} - JSON library
 \item\texttt{magnet.c} - modified magnet.c source code to pass POST data.
 \item\texttt{rad-check.lua} - Lua script for automating eapol\_test
 \item\texttt{rad-check.py} - deprecated python version
 \item\texttt{templates/} - wpa\_supplicant type templates for eapol\_test
\end{enumerate}
\item\texttt{security/} - PKI and certificates
\begin{enumerate}
 \item\texttt{CA/} - copy of /etc/pki/CA, containing CA certificate and private key,
                with a registry of certificate requests and issues 
 \item\texttt{secure/} - copy of /root/secure, x509 certificates and keys used during
                the testing of the system. Because the test system is still
                live, passphrases for keys are not provided.
\end{enumerate}
\end{enumerate}

\clearpage
\section{Report Allocation}
\label{sec:words}
\begin{center}
%\begin{table}
\begin{tabularx}{\linewidth}{|XXX|}
\hline
Group Member & report section\\ \hline
Thomas Grainger & Malware Lists\\
& Specification\\
& Customer Interaction\\
& Architectural Design\\
& Trend Analysis Design\\
& URL Descovery Design\\ \hline

Weike Liao & Literature Review \\
& Architectural Design \\
& Trend Analysis Implemenation \\
& URL Descovery Implemenation \\
& Wine \\
& ClamAV \\ \hline

Chris Orchard & Description \\
& Capture HPC \\
& Group Approach \\ \hline

Nafiseh Vahabi & Literature Review  \\
& Related Work \\
& Problem Definition \\
& Manual HTML \\
& Classifier \\
& Future Development \\ \hline
\hline
\end{tabularx}
%\end{table}
\end{center}